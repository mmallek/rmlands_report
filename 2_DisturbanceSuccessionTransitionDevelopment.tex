\chapter{Disturbance and Succession Transition Development}

\section{Succession Transitions}
Many cover types have a probabilistic transition between condition classes. As part of our process and of generating the description documents, I wrote a script to generate a survivorship curve for patches in a given condition class showing their gradual transition to the next stage of development or from open to closed characteristic.

\section{Disturbance Transitions}
The BPS models described MRI values for each type of fire that could affect a particular condition class. The fire types are high severity, mixed severity, and low severity. Each type of fire, and its FRI, can trigger a transition to a different condition class within the model or have no change at all.

I reinterpreted a significant amount of the BPS models.

An important characteristic of RMLands is that we reject the idea of high, mixed, and low severity fires. Instead, we recognize that all fires form mosaics of burned and unburned land within a given perimeter, and that the severity of these burns varies on a continuous scale. We remain constrained to categorical characterization of fire, but have eliminated mixed severity fires from the model. A mixed severity fire, as it is commonly thought of in land management, would be one fire in the model, with patches assigned to either high or low mortality levels probabilistically. The way I interpreted the BPS models for this particular task is that any fire, regardless of its initial characterization, resulted in a patch transitioning to early development, was designated a high mortality fire. All others were low mortality.

The MRIs given in the BPS models are not true MRIs, as in reality a condition class cannot have an MRI. Because the BPS models were derived from expert opinion and reviewed by experts as well, and are the only comprehensive and cohesive set of data on vegetation transitions in our study area, we elected to utilize them as much as possible. I took the inverse of each given MRI value to generate a probability that could then be compared across condition classes, predicted fire mortality, and vegetation types.
---
In the first iteration of this model (presented to the Tahoe NF in February 2014), we calculated an overall FRI for each cover type using VDDT/LANDFIRE values, modified with expert opinion consisting of input from our partners on the TNF. For the second iteration of the model, we have included separate FRIs for each condition class associated with a given cover type. These were calculated again based on VDDT/LANDFIRE. For a few common cover types, Hugh Safford provided us with modified VDDT inputs. The results [will be] shown to the Tahoe NF team for approval.

The FRIs are not directly the same as the probabilities as what VDDT uses. VDDT focuses on a type of fire (e.g. mixed) that leads to a particular outcome (e.g. resetting to early seral conditions). VDDT includes 'mixed severity' fire, a construct that we do not include in our model. Instead, we have two types of fire, judged based on their outcomes: high mortality events, and low mortality events. To interpret the VDDT models, I analyzed not only the fire type (replacement, mixed, or surface), but also the condition transitions that occurred as a consequence of that fire. I sorted the probabilities in the VDDT model into 3 bins. High mortality fires are those that result in conversion to early seral, regardless of whether they are called ``replacement'' or ``mixed.'' All other fires are considered low mortality. For closed systems, I separated probabilities associated with fires that led to no change in the current condition, versus conversion to an open condition. For open systems, I took VDDT at its word with respect to ``surface'' vs. ``mixed'' (since the outcome was the same for both). I summed the probabilities given for both low and high mortality fires, and for all fires. By taking the inverse, I arrived at fire return intervals, in years, for high, low, and any mortality events. Secondly, the proportion of high severity fires was calculated by dividing the probability of high mortality fire by the summed probability of fire. The full spreadsheet showing all calculations is `disturbance_prob_calcs.xlsx.'