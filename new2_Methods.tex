\chapter{Methods}
II. Methods
	A. Study Area
	B. Modeling Framework- RMLands (Kevin will write this)
	i. HRV
		1. Data prep
			B. Common Input Layers
				1. Input Layers: source, processing, purpose
		C. Model Parameterization
				1. State and Transition Models
				2. Disturbance parameters
					a. Climate
					b. Susceptibility
					c. Initiation
					d. Spread
					e. Mortality
		D. Model Calibration
		E. Model Execution
			3. Run parameters
				f. length/timesteps
		F. Data analysis
			1. Disturbance regime
			2. Vegetation response 
				a. Landscape composition (covcond plots)
				b. Landscape Structure/Patterns
	ii. Future Veg Treatments (match this to HRV header styles)
		1. Data prep
			B. Common Input Layers
				1. Input Layers: source, processing, purpose
		C. Model Parameterization
				1. State and Transition Models
				2. Disturbance parameters
					a. Climate
					b. Susceptibility
					c. Initiation
					d. Spread
					e. Mortality
		D. Model Calibration
		E. Model Execution
			3. Run parameters
				f. length/timesteps
			2. Post-processing (rescale, clip)
		F. Data analysis
			1. Disturbance regime
			2. Vegetation response 
				a. Landscape composition (covcond plots)
				b. Landscape Structure/Patterns

\section{Study Area}
\begin{figure}
\caption{Western United States. Sierra Nevada Ecoregion is highlighted in red. The project landscape is located near the northern extent of the Sierra Nevada.}
The Sierra Nevada is a significant North American mountain range, located in California east of the Central Valley and extending from Fredonyer Pass in the north to southern Kern County in the south. Much of the Sierra Nevada is reserved as federally-held public land, managed by the U.S. Forest Service, Bureau of Land Management, and the National Park Service. The Plumas and Tahoe National Forests are located in the northern portion of the Sierra Nevada. The project landscape (Figure 2) is located on the northern part of the Tahoe National Forest, on the Yuba River and Sierraville Ranger Districts, and comprises about 181,550 hectares. It is composed of a set of three HUC-5 watersheds, the Upper North Yuba River, the Middle Yuba River, and the Lower North Yuba River, referred to in this proposal as the Upper Yuba River watershed. 
The topography of the project landscape consists of rugged mountains incised by two major and a few minor river drainages. Elevation ranges from about 350 to 2500 meters. The area receives 30-260 cm of precipitation annual, most of which falls as snow in the mid to upper elevations (Storer and Usinger 1963). Some areas in the mid-elevation band receive high precipitation compared to the region, resulting in patches of exceptionally productive forest (Alan Doerr, pers. comm.). Vegetation is tremendously diverse and changes slowly along an elevational gradient and in response to local changes in drainage, aspect, and soil structure. Grasslands, chaparral, oak woodlands, mixed conifer forests, and subalpine forests are all found within the study area.

















